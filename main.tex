\documentclass{article}
\usepackage[utf8]{inputenc}
\usepackage{graphicx}
\usepackage{amsmath}
\usepackage{float}
\usepackage{lscape}
\usepackage{subcaption} %Side by side images
\usepackage[a4paper, margin=3cm]{geometry}

% Fancy tables
\usepackage{booktabs}

% Code viewing
\usepackage{listings}
\usepackage{color}

% Table of contents control: 
\usepackage{setspace}

\definecolor{mygreen}{rgb}{0,0.6,0}
\definecolor{mygray}{rgb}{0.5,0.5,0.5}
\definecolor{mymauve}{rgb}{0.58,0,0.82}

\lstset{ 
  backgroundcolor=\color{white},   % choose the background color; you must add \usepackage{color} or \usepackage{xcolor}; should come as last argument
  basicstyle=\footnotesize,        % the size of the fonts that are used for the code
  breakatwhitespace=false,         % sets if automatic breaks should only happen at whitespace
  breaklines=true,                 % sets automatic line breaking
  captionpos=b,                    % sets the caption-position to bottom
  commentstyle=\color{mygreen},    % comment style
  deletekeywords={...},            % if you want to delete keywords from the given language
  escapeinside={\%*}{*)},          % if you want to add LaTeX within your code
  extendedchars=true,              % lets you use non-ASCII characters; for 8-bits encodings only, does not work with UTF-8
  firstnumber=1,                   % start line enumeration with line 1000
  frame=single,	                   % adds a frame around the code
  keepspaces=true,                 % keeps spaces in text, useful for keeping indentation of code (possibly needs columns=flexible)
  keywordstyle=\color{blue},       % keyword style
  language=Octave,                 % the language of the code
  morekeywords={*,...},            % if you want to add more keywords to the set
  numbers=left,                    % where to put the line-numbers; possible values are (none, left, right)
  numbersep=5pt,                   % how far the line-numbers are from the code
  numberstyle=\tiny\color{mygray}, % the style that is used for the line-numbers
  rulecolor=\color{black},         % if not set, the frame-color may be changed on line-breaks within not-black text (e.g. comments (green here))
  showspaces=false,                % show spaces everywhere adding particular underscores; it overrides 'showstringspaces'
  showstringspaces=false,          % underline spaces within strings only
  showtabs=false,                  % show tabs within strings adding particular underscores
  stepnumber=1,                    % the step between two line-numbers. If it's 1, each line will be numbered
  stringstyle=\color{mymauve},     % string literal style
  tabsize=2,	                   % sets default tabsize to 2 spaces
  title=\lstname                   % show the filename of files included with \lstinputlisting; also try caption instead of title
}


%User defined commands
\newcommand*\mean[1]{\overline{#1}}

\newcommand*\hadamarddiv{\obslash} %Hadamard division

\title{Project-thesis}
\author{Erik Rundhovde M\o renskog}
\date{September 2019}

% Document setup
\setlength{\parindent}{0em}
\setlength{\parskip}{1em}

\begin{document}

% Title page
\begin{titlepage}
    
    \includegraphics[width=0.4\textwidth]{figures/ntnu_hovedlogo_eng_svart.png}
    
    \vspace{1cm}
    \huge
    
    \textbf{Project Thesis TFE4580}
 
    \vspace{0.5cm}
    
    Sensor Fusion \\ 
    Camera and Spectrometer
 
    \vspace{1.5cm}
    \Large
    
    \textbf{Erik Rundhovde M\o renskog}

    \vfill

    December 2019
    
    \vspace{0.5cm}

    Supervised by Harald Ian Muri and Dag Roar Hjelme
\end{titlepage}

\singlespacing
\tableofcontents

\section{Abstract}
The waste industry is in need of cheaper ways of analyzing waste. This paper will look into if it is feasible to sensor fuse camera and spectrometer as an alternative to hyperspectral cameras. This feasibility is tested by comparing average values that should be proportional too each other between the systems. If these values are proportional to each other it will make the calibration of camera to spectrometer easier. 

\section{Introduction}
Hyperspectral imaging is extremely useful for classifying substances from a distance, but it is also very expensive. This paper will study if it is possible to correlate the data between a camera and a spectrometer. This will be a step towards making it feasible to use the much cheaper combination of camera and spectrometer instead of a hyperspectral camera. The advantage of a hyperspectral camera is that you can get high spectral resolution in combination with high spatial resolution, usually 1 dimension for each. The proposed alternative is one where a camera is combined with a spectrometer. The camera will be able to give high spatial resolution in 2D, while the spectrometer will provide high spectral information for the same sample. This could be a good replacement in ideal situations with homogenous objects, or if the average characteristics of a heterogenous substance is of interest. The theoretical setup is shown in figure \ref{fig:measurement_setup}. 

The postulate to be explored: 
The spatial average across an image of each color, is comparable with the spectral average of a spectrum. That is if that spectrum has been multiplied with the quantum efficiency of each color of the camera sensor. 

Postulates further that the relationship between the values can be found through linear regression. This will have the following uses: 
\begin{itemize}
    \item The regression line can be used to tell if an image and a specter is taken correctly, that they are watching the same object. 
    \item If the previous is known to be true the regression line can give information about noise, from ambient light or other sources. 
\end{itemize}

%TODO: Add the focus which is finding correlating values between camera and spectrometer

\begin{figure}[h]
    \centering
    \includegraphics[width=0.5\textwidth]{figures/pt_setup.pdf}
    \caption{Measurement setup}
    \label{fig:measurement_setup}
\end{figure}



\section{Theory}
%TODO: Need theory about ccd? It is not really relevant is it?
%Need theory about the different uses of the same sensor in spectrometer and camera, I need this defend that the values should have an correlation. 

\subsection{Camera}
Cameras are widely used to photograph objects, and works by focusing light onto a photo-sensor. The camera used in this thesis uses a Charged-coupled device (CCD) photo-sensor \cite{JYIVolumeThree}.

\subsection{Spectrometry}
Spectrometer is a widely used tool for analyzing substances. It is a remote sensing tool that give good spectral information about the light that enters the fiber. As can be seen in figure \ref{fig:spectrometer_inside} the light is split into the different wavelengths before hitting the CCD sensor. This has the effect that the sensor reads spectral information instead of spatial as a camera would. 

\begin{figure}[h]
    \centering
    \includegraphics[width=0.5\textwidth]{figures/Mini-spectrometer-open-bench.png}
    \caption{Inside of a spectrometer \cite{KAI0340640480}}
    \label{fig:spectrometer_inside}
\end{figure}

Because of this it can be used to give an average spectrum of the area under the acceptance cone of the fiber. 
%TODO3 Add theory about how the spectrometer works. I could maybe skip this point as it is not necessary for explaining the correlation, but I'm afraid it is necessary for a rigorous description of why the correlation makes sense. Same for the camera

\subsection{Reflection}
\label{sec:theory_reflection}
Reflection of light describes the notion of light hitting materials and getting a change of path due to the exchange of energy with the material. There are two extremes when we talk about reflection:

\textbf{Specular reflection} denotes the case where the light is reflected in a unison manner from the sample, all in one direction. This concept is shown in figure \ref{fig:specular_reflection}.

\begin{figure}[h!]
    \centering
    \includegraphics[width=0.5\textwidth]{figures/theory/Specular-Reflection.png}
    \caption{Example of specular reflection \cite{SpecularReflectionOcean}}
    \label{fig:specular_reflection}
\end{figure}

\textbf{Diffusive reflection} denotes the case where the light is scattered in every direction. This concept is shown in figure \ref{fig:diffusive_reflection}.

\begin{figure}[h!]
    \centering
    \includegraphics[width=0.5\textwidth]{figures/theory/Diffuse-Reflection.png}
    \caption{Example of diffusive reflection \cite{DiffuseReflectionOcean}}
    \label{fig:diffusive_reflection}
\end{figure}

Both of these reflection types are ideal and all real reflection will be a combination of these two. Not even the best available mirrors exhibit perfect specular properties, and no material scatters light equally in all directions. It is however useful to have two extremes to compare the results too. 

%TODO: Im not sure yet about including these
%\subsubsection{Specular} 
%\subsubsection{Diffusive}
%\subsection{Relative reflection}

\subsection{Noise and dark current}
\label{sec:noise_and_dark_current}
Both the images and the spectrums will have noise in them, making the readings less accurate. One type of noise that is a problem in spectrometry is noise from dark currents. These are currents that will be detected in the CCD sensor even though there is no incoming light. 



%TODO: Should add:
% Linear regression
% Colorimetry
% 

\section{Definitions}
This section will introduce all the math notation and equations that will be used in the paper. 

\subsection{Linear regression}
\label{sec:regression}
Linear regression is a technique to find the function of the form (\ref{eq:linear_function}) that have the minimum total square distance to the points in the set. 

\begin{equation}
    f(x) = ax + b
    \label{eq:linear_function}
\end{equation}


\subsection{Image}

Each pixel of the image will be described by a vector (\ref{eq:vector_pixel_bgr}), where each point on the vector describes one of the colors blue (B), green (G) and red (R).  This together gives the color combination of each pixel. Each color is an unsigned integers. 


\begin{equation}
    \label{eq:vector_pixel_bgr}
    \vec{p}_{ij} = [B,G,R]
\end{equation} 

One whole image will be stored in a matrix where each element of the matrix is a vector described by (\ref{eq:vector_pixel_bgr}). The matrix $A_{N\cdot M}$, where $N$ is the number of rows and $M$ is the number of columns, is shown in (\ref{eq:image_matrix})

\begin{equation}
    \label{eq:image_matrix}
    A = A_{N\cdot M} =  
    \begin{bmatrix}
        \vec{p}_{11} & \vec{p}_{12} & \cdots & & \vec{p}_{1M}  \\
        \vec{p}_{21} & \ddots &        &       &                \\
        \vdots       &        &\vec{p}_{ij}&   & \vdots          \\
                     &        &        & \ddots&                  \\
        \vec{p}_{N1} &        &        &       & \vec{p}_{NM}  
    \end{bmatrix}
\end{equation}

A second matrix to introduce is the special case where the image is the reference image, this special case will be denoted as $A_0$.

\subsubsection{Spatial average}
\label{sec:spatial_average}

An important equation later is the one that takes the spatial average of the matrix, this is done with (\ref{eq:spatial_sum}).
\begin{equation}
    \label{eq:spatial_sum}
    \mean{\vec{p}} = \frac{1}{NM} \sum_{i=0}^{N-1} \sum_{j=0}^{M-1} \vec{p}_{ij}
\end{equation}


\subsubsection{Hadamard division}
Element wise division will be used to compare two pictures, and is defined by the Hadamard division \cite{HadamardDivisionInfixed}:
\begin{equation}
    \label{eq:element_wise_division_image}
    A \oslash  A_0  \frac{A_{ij}}{A_{0ij} } %TODO
\end{equation}


\subsection{Spectrum}
\label{sec:spectrum}

The spectrum read from the spectrometer is saved in a $C x 2$ matrix, where $C$ is the dataset length, the first column is the wavelength ($\lambda$) and the second column is the corresponding intensity (\ref{eq:intensity})
\begin{equation}
    \label{eq:intensity}
    I = I(\lambda)    
\end{equation}

\begin{equation}
    \label{eq:intensity_0_background}
    I_0 = I_0(\lambda)
\end{equation}

We will also define a value (\ref{eq:intensity_0_background}) which is $I$ for the special case where the spectrum is muffin the background spectral response without any objects. From these definitions we define relative reflectance $RR$ (\ref{eq:relative_reflectance}). 

\begin{equation}
    \label{eq:relative_reflectance}
    RR = \frac{I}{I_0}
\end{equation}


The introduction of (\ref{eq:relative_reflectance_minus_one}) will make it easier see the changes in the spectrum when multiplying with QE.  
\begin{equation}
    \label{eq:relative_reflectance_minus_one}
    RR_2 = RR - 1
\end{equation}

We further introduce the notion of finding the spectrum corresponding to one color in the camera. Each pixel in the camera measures the light intensity for blue, green and red with a certain quantum efficiency given by the manufacturer. These values will be represented in a vector (\ref{eq:quantum_efficiency}) with three values corresponding to each wavelength $\lambda$. These values will represent how well each color is received by the camera and will be a float between zero and 1. 

\begin{equation}
    \label{eq:quantum_efficiency}
    \vec{QE}(\lambda)    
\end{equation}

This value can theoretically be used to relate the relative picture values with the relative reflectance values. This would be a major advantage as it can give us an insight into the noise factor affecting the sensor fusion.  

\subsubsection{Sectral average}
\label{sec:spectral_average}

The spectral equivalent of spatial sum (\ref{eq:spatial_sum}) is to take the integral of the graph and divide it by the wavelength range (\ref{eq:average_integral}). The trapezoidal rule was used to approximate the integral \cite{TrapezoidRuleMathematical}. 

\begin{equation}
    \label{eq:average_integral}
    \mean{\vec{RR}} = \frac{1}{\lambda_1 - \lambda_0} \int_{\lambda_0}^{\lambda_1} RR \cdot \vec{QE} \,\mathrm{d}\lambda 
\end{equation}


\subsubsection{Error function}
\label{sec:error_function}
To have an idea of how well an estimation works, an error function can be used. The function that will be used here takes the Euclidean distance between a data-point and the closest point in the estimation. The estimation will be a straight line in this setup, so the minimum distance between a point and the line can be found using the cross product. This function works by using three points, two of them should be points on the line ($p_0$ and $p_1$), and one should be the data point ($p$). The function is given in (\ref{eq:error_function}), where $\left\lVert .\right\rVert$ shows the second norm. 

\begin{equation}
    \label{eq:error_function}
    d = \frac{(p - p_0) \times (p_1 - p_0)}{\left\lVert p_1 - p_0\right\rVert _2}
\end{equation}

\section{Method}
Both the spectrums and the images will be analysed using spectrometer methods, this will open up for the opportunity to pre-process spectral and spatial data in the same way. We will then hopefully be able to compare the data more directly. The spectrometer processing method we want to use is relative reflectance ($RR$). Since this method is also being used with the spectrometer, it will be denoted $RR'$ whenever it is used for imaging. As can be seen from (\ref{eq:relative_reflectance}), relative reflectance is done by dividing the interesting values with a reference. The translation to image processing must be to divide every image pixel with a pixel value from a reference image. For comparing values with the spectrometer it is fully ok to do this division as long as the reference image is non-zero for all pixels. For that reason it is important with a well lit and preferably white background. 


One problem with this method is that images can only be visualized as three sets of integers between 0 and 255, i.e. the colors blue, green and red. Therefore we divide the division into to two cases when visualizing; reference divided by image and image divided by reference. We will denote the product $RR'_{negative}$ for the reference divided by image case, and $RR'_{positive}$ for the image divided by reference case. The processing can be view in figure \ref{fig:image_visualization_program_flow} 

\begin{figure}[h]
    \centering
    \includegraphics[width=0.5\textwidth]{figures/image_program_flow.pdf}
    \caption{Image visualization process flow}
    \label{fig:image_visualization_program_flow}
\end{figure}


\subsection{Spectrum processing}
\label{sec:spectrum_processing}

\begin{figure}[h]
    \centering
    \includegraphics[width=0.5\textwidth]{figures/thesis_program_flow.pdf}
    \caption{Spectrum process flow}
    \label{fig:spectrum_process_flow}
\end{figure}

\subsection{Correlating Spectrometer to Camera}
\label{sec:method_correlating_spectrum_to_camera}
To get an idea of how well calibrated the camera is to the spectrometer and vice versa I propose the following calculation: 
Take the spatial average across the image from the camera and divide it with the spectral average of the spectrometer. 

\begin{equation}
    \label{eq:correlating_spectrum_to_camera}
    K = \frac{\mean{\vec{p}}}{\mean{RR}}
\end{equation}

This process is shown in figure \ref{fig:correlating_spectrum_and_image}.

\begin{figure}[h]
    \centering
    \includegraphics[width=0.75\textwidth]{figures/image_comparison_with_spectrometer.pdf}
    \caption{Correlating value between image and spectrum}
    \label{fig:correlating_spectrum_and_image}
\end{figure}


\subsection{Light}
Light is a crucial part of this project, as it is the source of input for both the camera and the spectrometer. It's also the link between the two sensors. The choosing of a sensor that can support both sensor types is therefore paramount. The camera is less selective on the spectral properties of the light source as it only requires a light source that is approximately white, i.e. have similar amounts of red, green and blue "wavelengths". It is however more selective in the spatial region as it can arise more problems for the camera if the lighting creates a lot of shadows or local problems like strong specular reflection making the pixel go into saturation. As implied the spectral properties of the light is more important for the spectrometer. For the spectrometer we want the spectrum to be as flat as possible. 

The characterization of the light source will be based on considerations from \cite{martinPracticalGuideMachine}, but also unfortunately be limited by available sources at the lab. % This paper provides a longer checklist, that can be simplified greatly under the following conditions: Stationary objects,


\section{Results}

The spatial and spectral averages are plotted against each other in figure \ref{fig:spectral_vs_spatial_values}. 
\begin{figure}[h]
    \centering
    \includegraphics[width=1\textwidth]{Plots/spectral_vs_spatial_average.png}
    \caption{Spatial and spectral averages plotted against each other}
    \label{fig:spectral_vs_spatial_values}
\end{figure}



\section{Discussion}
The spatial and spectral averages should be codependent. This is because we have the same type of sensor, that are imaging the area in two different ways, but through taking the average to eliminate that difference it should amount to proportional values. 

In figure \ref{fig:spectral_vs_spatial_values} a linear relation can be seen for each of the three colors. This means that they can be represented on the form described in section \ref{sec:regression}. It further looks like each of the color sensors have approximately the same number $a$, meaning that they have the similar derivatives, but different $b$ means that they cross the y axis at different points. 

\section{Conclusion}

It is a clear correlation between spatial and spectral average which seems to follow the regression lines well. There is however an amount of error that makes it hard to use for determining if the image and spectrum is perfectly of the same object. It is very good for finding out if there is to much noise from the ambient light. It seems like the test was lacking somewhat in diversity.

\bibliographystyle{plain}
\bibliography{references.bib}

\section{Appendix}


\subsection{Code}
\label{sec:appendix_code}
All the code used in this project is contained within one file. The main function has all the program flow, while the functions introduced above are written in a functional programing manner. 

Code used in this project: 

\lstinputlisting[language=Python]{image_and_spectrum_analyzis.py}

\subsection{Plots not included in the report}
\label{sec:appendix_plots}

Figure \ref{fig:quantum_efficiency_camera} shows the quantum efficiency of the Bayer mask in the cameras CCD sensor. 
\begin{figure}[h]
    \centering
    \includegraphics[width=1\textwidth]{Plots/quantum_effiency.png}
    \caption{The quantum efficiency of the camera}
    \label{fig:quantum_efficiency_camera}
\end{figure}


Figure \ref{fig:relative_reflection_around_zero} shows $RR-1$ for nine of the objects being analyzed. 
\begin{landscape}
\begin{figure}[t]
    \centering
    \includegraphics[width=1\paperwidth]{Plots/relative_reflectance_around_zero_with_qe_color_response.png}
    \caption{Relative reflection centered around zero}
    \label{fig:relative_reflection_around_zero}
\end{figure}
\end{landscape}




\end{document}
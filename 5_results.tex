\section{Results}

The spatial and spectral averages are plotted against each other in figure \ref{fig:spectral_vs_spatial_values}. 
\begin{figure}[h]
    \centering
    \includegraphics[width=1\textwidth]{Plots/spectral_vs_spatial_average.png}
    \caption{Spatial and spectral averages plotted against each other}
    \label{fig:spectral_vs_spatial_values}
\end{figure}



\section{Discussion}
The spatial and spectral averages should be codependent. This is because we have the same type of sensor, that are imaging the area in two different ways, but through taking the average to eliminate that difference it should amount to proportional values. 

In figure \ref{fig:spectral_vs_spatial_values} a linear relation can be seen for each of the three colors. This means that they can be represented on the form described in section \ref{sec:regression}. It further looks like each of the color sensors have approximately the same number $a$, meaning that they have the similar derivatives, but different $b$ means that they cross the y axis at different points. 
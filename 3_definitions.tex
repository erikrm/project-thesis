\subsection{Definitions}
This subsection will introduce all the math notation and equations that will be used in the paper. 

\subsubsection{Linear regression}
\label{sec:regression}
Linear regression is a technique to find the function of the form (\ref{eq:linear_function}) that have the minimum mean square distance to the points in the set. 

\begin{equation}
    f(x) = ax + b
    \label{eq:linear_function}
\end{equation}

\subsubsection{Image represenation}

Each pixel of the image will be described by a vector (\ref{eq:vector_pixel_bgr}), where each point on the vector describes one of the colors blue (B), green (G) and red (R).  This together gives the color combination of each pixel. Each color is an unsigned integers. 


\begin{equation}
    \label{eq:vector_pixel_bgr}
    \vec{p}_{ij} = [B,G,R]
\end{equation} 

One whole image will be stored in a matrix where each element of the matrix is a vector described by (\ref{eq:vector_pixel_bgr}). The matrix $A_{N \times M}$, where $N$ is the number of rows and $M$ is the number of columns, is shown in (\ref{eq:image_matrix})

\begin{equation}
    \label{eq:image_matrix}
    A = A_{N \times M} =  
    \begin{bmatrix}
        \vec{p}_{11} & \vec{p}_{12} & \cdots & & \vec{p}_{1M}  \\
        \vec{p}_{21} & \ddots &        &       &                \\
        \vdots       &        &\vec{p}_{ij}&   & \vdots          \\
                     &        &        & \ddots&                  \\
        \vec{p}_{N1} &        &  \cdots &       & \vec{p}_{NM}  
    \end{bmatrix}
\end{equation}

A second matrix to introduce is the special case where the image is the reference image, this special case will be denoted as $A_0$.

\subsubsection{Spatial average}
\label{sec:spatial_average}

An important equation later is the one that takes the spatial average of the matrix, this is done with (\ref{eq:spatial_sum}).
\begin{equation}
    \label{eq:spatial_sum}
    \mean{\vec{p}} = \frac{1}{N \cdot M} \sum_{i=0}^{N-1} \sum_{j=0}^{M-1} \vec{p}_{ij}
\end{equation}

\subsubsection{Hadamard product}
\label{sec:hadamard_product}
Element wise multiplication will be used to combine the quantum efficiency (QE) with the spectrum, it is defined by the Hadamard product \cite{millionIntroductionBasicResults} and shown in ()
\begin{equation}
    \label{eq:element_wise_product_spectrum}
    RR \otimes QE \doteq RR_i \cdot QE_i
\end{equation}

\subsubsection{Hadamard division}
\label{sec:hadamard_division}
Element wise division will be used to compare two pictures, and is defined by the Hadamard division \cite{HadamardDivisionInfixed} and shown in (\ref{eq:element_wise_division_image}).
\begin{equation}
    \label{eq:element_wise_division_image}
    A \oslash  A_0 \doteq  \frac{A_{ij}}{A_{0ij} } %TODO
\end{equation}


\subsubsection{Spectrum}
\label{sec:spectrum}

The spectrum read from the spectrometer is saved in a $C \times 2$ matrix, where $C$ is the dataset length, the first column is the wavelength ($\lambda$) and the second column is the corresponding intensity (\ref{eq:intensity})
\begin{equation}
    \label{eq:intensity}
    I = I(\lambda)    
\end{equation}

\begin{equation}
    \label{eq:intensity_0_background}
    I_0 = I_0(\lambda)
\end{equation}

We will also define a value (\ref{eq:intensity_0_background}) which is $I$ at the special case where the spectrum is the background spectral response without any objects. From these definitions we define relative reflectance $RR$ (\ref{eq:relative_reflectance}). 

\begin{equation}
    \label{eq:relative_reflectance}
    RR = \frac{I}{I_0}
\end{equation}

The introduction of (\ref{eq:relative_reflectance_minus_one}) will make it easier see the changes in the spectrum when multiplying with QE. It will make the wavelengths that are equal in the object-picture and the reference-picture be zero, while showing wavelengths that have gotten stronger as positive numbers and negative numbers for weaker wavelengths.  

\begin{equation}
    \label{eq:relative_reflectance_minus_one}
    RR_2 = RR - 1
\end{equation}

We further introduce the notion of finding the spectrum corresponding to one color in the camera. Each pixel in the camera measures the light intensity for blue, green and red with a certain quantum efficiency given by the manufacturer. These values will be represented in a vector (\ref{eq:quantum_efficiency}) with three values corresponding to each wavelength $\lambda$. These values will represent how well each color is received by the camera and will be a float between zero and 1. 

\begin{equation}
    \label{eq:quantum_efficiency}
    \vec{QE}(\lambda)    
\end{equation}

This value can theoretically be used to relate the relative picture values with the relative reflectance values. This would be a major advantage as it can give us an insight into the noise factor affecting the sensor fusion.  

\subsubsection{Spectral average}
\label{sec:spectral_average}

The spectral equivalent of spatial sum (\ref{eq:spatial_sum}) is to take the integral of the graph and divide it by the wavelength range (\ref{eq:average_integral}). The trapezoidal rule will be used to approximate the integral \cite{TrapezoidRuleMathematical}. 

\begin{equation}
    \label{eq:average_integral}
    \mean{\vec{RR}} = \frac{1}{\lambda_1 - \lambda_0} \int_{\lambda_0}^{\lambda_1} RR \cdot \vec{QE} \,\mathrm{d}\lambda 
\end{equation}

\subsubsection{Error function}
\label{sec:error_function}
To have an idea of how well an estimation works, an error function can be used. The function that will be used here takes the Euclidean distance between a data-point and the closest point in the estimation. The estimation will be a straight line in this setup, so the minimum distance between a point and the line can be found using the cross product. This function works by using three points, two of them should be points on the line ($p_0$ and $p_1$), and one should be the data point ($p$). The function is given in (\ref{eq:error_function}), where $\left\lVert .\right\rVert$ denotes the second norm. 

\begin{equation}
    \label{eq:error_function}
    d = \frac{(p - p_0) \times (p_1 - p_0)}{\left\lVert p_1 - p_0\right\rVert _2}
\end{equation}